\documentclass[../../Rapport RayTracer]{subfiles}

\begin{document}

La méthode retenu pour effectuer le parsing d'un fichier POV est celle de l'automate à états finis. C'est donc pour cela que l'on a crée la classe Automat. Le principe est simple, on associe à chaque figure une classe d'état. Une classe EtatSphere, une classe EtatTriangle, etc. Chaque classe d'état s'occupe de parser son objet associé. La classe Automat s'occupe juste d'appeler au bon moment les états, c'est à dire quand le jeton de son StreamTokenizer rencontre un mot connu, soit sphère, plane, box... Les différents états possibles de l'automate sont définis dans l'énumération State. Concrètement, dans la classe Automat, on traverse le fichier jusqu'à ce qu'on rencontre une figure connu. A ce moment là, on fixe l'état sur cette figure et on appelle la classe d'état correspondante. Dès lors que la parsing de la figure est fini, on regarde le prochain mot et on fixe l'état correspondant. Tout ces traitements dont effectués dans une boucle qui a pour condition de d'arrêt la fin du fichier. Ensuite on switch sur l'état courant et on définit autant de case qu'il y a d'états. On a aussi définit un état vide, l'état OUTSIDE qui correspond à aucun élément de syntaxe. Dès lors que l'on se trouve dans un état OUTSIDE, on avance juste notre jeton dans le fichier pour tomber sur la prochaine figure, s'il y en a une.

 \lstinputlisting[language=Java]{../../code/automat.java}
 
De plus nous avons implémenté des sous états qui se trouvent dans chaque classe. Ils sont contenus dans des énumérations présentes dans chaque classe d'états. Ces états correspondent en fait à des éléments de syntaxe que l'on trouve dans un bloc de figure

\end{document}

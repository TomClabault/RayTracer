Tout en respectant le sujet, nous avons décidé d'implémenter un certain nombre de fonctionnalités facultative, mais qui nous semblait intéréssante.

Nous avons donc réalisé, en plus de ce qui était exigé par le sujet :
\begin{itemize}
    \item La gestion du rendu sur plusieurs threads.
    \item Le déplacement dans la scène 3d à l'aide de touches du clavier (avec actualisation de l'affichage en conséquence).
    \item le choix de la taille du rendu ainsi que son étirement possible pour conserver de la fluidité (mode automatique)
    \item L'usage d'une skybox.
    \item Les objets 3d possédant de la rugausité (raughness).
    \item L'ouverture et l'enregistrement du rendu 3D à l'aide de l'interface graphique.
    \item Le changement des parametres du rendu à la vollée (avec actualisation du rendu en conséquence).
    \item Un mode automatique permettant l'étirement propre de la fenêtre de rendu.
    \item L'ajout d'une texture en damier générée mathématiquement.
\end{itemize}

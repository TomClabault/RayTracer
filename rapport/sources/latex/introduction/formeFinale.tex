Tout en respectant le sujet, nous avons décidé d'implémenter un certain nombre de fonctionnalités facultatives, mais qui nous semblaient intéressantes.

Nous avons donc réalisé, en plus de ce qui était exigé par le sujet :
\begin{itemize}
    \item La calcul du rendu sur plusieurs threads
    \item Le déplacement dans la scène 3D à l'aide de touches du clavier (avec actualisation de l'affichage en conséquence)
    \item le choix de la taille du rendu ainsi que son étirement à la taille de l'écran si souhaité (mode automatique)
    \item L'utilisation possible d'une skybox
    \item Le support de la rugosité des matériaux (roughness)
    \item L'enregistrement du rendu sur le disque à l'aide de l'interface graphique
    \item Le changement des paramètres du rendu à la volée (avec actualisation du rendu en conséquence) au moyen de la "Toolbox"
    \item L'ajout d'une texture de damier générée mathématiquement
	\item Le support de plusieurs sources de lumière
	\item L'implémentation d'anti-crénelage afin de réduire les effets visuels "d'escaliers"
\end{itemize}

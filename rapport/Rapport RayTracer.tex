\documentclass[11pt]{article}
\usepackage [french]{babel}
\usepackage [T1]{fontenc}

\usepackage[linesnumbered, ruled, french, onelanguage]{algorithm2e}
\usepackage{adjustbox}%Permet de centrer les figures dans la largeur de la page même si les figures sont plus larges que \textwidth
\usepackage{amssymb}
\usepackage{amsmath}
\usepackage[toc,page,title,titletoc,header]{appendix}
\usepackage{expl3}%Pour la control sequence /ExlpSyntaxOn demandée par l'utilisation de subfiles apparemment...
\usepackage{gensymb}%pour pouvoir écrire le signe °
\usepackage{geometry}%Pour changer la largeur des marges du document notamment
\usepackage{graphicx}
\usepackage{hyperref}%pour les liens dans la bibliographie
\usepackage{listings}
\usepackage{placeins}%pour utiliser FloatBarrier afin que les figure respectent bien leur position dans le code
\usepackage{slashbox}%Case séparée en deux tout en haut à gauche des tableaux à double entrées
\usepackage{stmaryrd}%pour les crochets à double barres d'intervalles de nombre entiers
\usepackage{tikz}
\usepackage{xcolor}%/definecolor et /color
\usepackage{subfiles}

\usepackage{etoolbox}%pour /AtBeginEnvironment
\AtBeginEnvironment{appendices}{\renewcommand{\thesection}{\Alph {section}}}%Pour recommencer à compter les sections à 0 en rentrant dans l'annexe et pour compter avec des lettres et non des chiffres
\renewcommand{\appendixpagename}{\centering Annexes}%Pour centrer le titre de la partie annexe
\renewcommand{\appendixtocname}{Table des annexes} % Pour faire apparaître les annexes dans la table of contents
\setlength{\parskip}{2mm}%Pour mettre de l'espacement entre les paragraphes

%%%%%%%%%%%%%% Couleurs de: https://texblog.org/2011/06/11/latex-syntax-highlighting-examples/ %%%%%%%%%%%%%%
\definecolor{javared}{rgb}{0.6,0,0} % for strings
\definecolor{javagreen}{rgb}{0.25,0.5,0.35} % comments
\definecolor{javapurple}{rgb}{0.5,0,0.35} % keywords
\definecolor{javadocblue}{rgb}{0.25,0.35,0.75} % javadoc

\lstset
{
language=Java,
keywordstyle=\color{javapurple}\bfseries,
stringstyle=\color{javared},
commentstyle=\color{javagreen},
morecomment=[s][\color{javadocblue}]{/**}{*/},
numbers=left,
numberstyle=\tiny\color{black},
stepnumber=1,
numbersep=10pt,
tabsize=4,
showspaces=false,
showstringspaces=false}
%%%%%%%%%%%%%% Couleurs de: https://texblog.org/2011/06/11/latex-syntax-highlighting-examples/ %%%%%%%%%%%%%%





\author{Baptiste CLOCHARD \\
    Tom CLABAULT\\
    Thibaut LETOURNEUR\\
    Willie TANZIL}
\title{Ray-Tracer}
\date{}
\geometry{hmargin=3cm, vmargin=2cm}

\begin{document}
\maketitle
\newpage
\tableofcontents
\newpage

\maketitle

\section{Introduction}

    \subsection{Présentation du sujet}
        Le sujet que nous avons choisi pour ce projet est "Rendu 3D par lancer de rayon".
Le lancer de rayon (raytracing en anglais) est la simulation du comportement des rayons de lumière visant à créer des images 3D réalistes.
Ce sujet nous demandait d'implémenter :
\begin{itemize}
    \item Le parsing d'un fichier .pov
    \item Les réflexions des matériaux
    \item Les réfractions 
    \item Un ombrage de Phong
\end{itemize}
Nous avons réussi à implémenter tout ce qui était demandé par le sujet.

    \subsection{Forme finale du projet}
        Tout en respectant le sujet, nous avons décidé d'implémenter un certain nombre de fonctionnalités facultatives, mais qui nous semblaient intéressantes.

Nous avons donc réalisé, en plus de ce qui était exigé par le sujet :
\begin{itemize}
    \item La calcul du rendu sur plusieurs threads
    \item Le déplacement dans la scène 3D à l'aide de touches du clavier (avec actualisation de l'affichage en conséquence)
    \item le choix de la taille du rendu ainsi que son étirement à la taille de l'écran si souhaité (mode automatique)
    \item L'utilisation possible d'une skybox
    \item Le support de la rugosité des matériaux (roughness)
    \item L'enregistrement du rendu sur le disque à l'aide de l'interface graphique
    \item Le changement des paramètres du rendu à la volée (avec actualisation du rendu en conséquence) au moyen de la "Toolbox"
    \item L'ajout d'une texture de damier générée mathématiquement
	\item Le support de plusieurs sources de lumière
	\item L'implémentation d'anti-crénelage SSAA (Super Sampling Anti-Aliasing) afin de réduire les effets visuels "d'escaliers"
\end{itemize}

    \subsection{Répartition du travail}
        Assez tôt dans le projet le travail a été réparti et est resté pratiquement inchangé. Au final, voici les contributions de chacun :
\begin{description}
    \item [Tom] s'est chargé des structures mathématiques, des matériaux, des calculs du raytracing et de la gestion des threads
    \item [Baptiste] s'est occupé des réfractions, de la gestion et de l'affichage du rendu, du déplacement de la caméra et des possibilités offertes par l'interface graphique.
    \item [Thibaut] a codé le parser de fichier pov à l'aide d'un automate ainsi que les scripts de "déploiement".
    \item [Willie] a réalisé les différents objets 3D (formes géométriques, lumière).
\end{description}

Chaque membre du groupe a été responsable de la javadoc des classes qu'il a codées.

Pour le rapport, chaque personne a détaillé les parties de son code et Baptiste s'est chargé des parties communes (introduction ...).

Pour information, les commits sur la forge au nom de DRUMOND sont ceux de Baptiste Clochard et ceux au nom de 21900403 sont ceux de Thibaut Letourneur.

Chaque étudiant à également aidé les autres lorsque le besoins s'en faisait sentir et ce durant les heures de tp ou non.


%\section{Mode d'emploi}
%\subsection{L'usage des scripts}

\subsection{L'utilisation de l'interface graphique}

\subsubsection{La fenêtre de choix de taille du rendu}

Lors de l'exécution du main de MainApp on obtient la fenêtre suivante :
\begin{figure}[h]
   \caption{Fênetre de sélection de la taille du rendu}
   \begin{center}
       \includegraphics{img/render.javafx/choiceWindow.jpg}
   \end{center}
\end{figure}

Cette fenêtre permet à l'utilisateur de sélectionner la résolution du rendu qu'il souhaite.
Activer le mode automatique maximisera la fenêtre et étendra le rendu sur tout l'écran. Si la résolution sélectionnée est alors inférieure à la résolution de l'écran, la qualité de l'image pourrait s'en trouver altérée.

\subsubsection{La fenêtre d'affichage du rendu}

Une fois que l'utilisateur a entré des paramètres valides, le rendu s'affichera.

Deux fenêtres s'afficheront alors :

\begin{figure}[h]
   \caption{Fenêtre du rendu}
   \begin{center}
       \includegraphics{img/render.javafx/render.jpg}
   \end{center}
\end{figure}

\begin{figure}[h]
   \caption{Fenêtre de la boite à outils}
   \begin{center}
       \includegraphics{img/render.javafx/toolbox.jpg}
   \end{center}
\end{figure}

%a déplacer dans readMe.txt

\section{Structuration du projet}
        \subsection{Le package povParser}
        \subsection{Les package liés au Ray Tracer}
            \subfile{latex/packages/RTpackages.tex}

        \subsection{Le package multithreading}
	  \subfile{latex/packages/multithreading.tex}

	\subsection{Le package geometry}
	    \subfile{latex/packages/geometry.tex}

        \subsection{Le package materials}
            \subfile{latex/packages/materials.tex}

        \subsection{Le package render}
        	Le package render contient le code de l'interface graphique, c'est-à-dire celui permettant l'affichage du rendu et des fenêtres.

\begin{description}
    \item [MainApp] contient le main de l'application et lance toutes les fenêtres.
    \item [SetSizeWindow] est la classe responsable de la fenêtre du choix de taille du rendu.
    \item [ImageWriter] est la classe responsable de l'affichage de la fenêtre du rendu.
    \item [Toolbox] est la classe responsable de la fenêtre de changement lors de l'affichage du rendu.
    \item [CameraTimer] est la classe responsable du déplacement de la caméra lors de l'appui de touches.
    \item [Window Timer et CounterFPS] sont les classes responsables de l'exécution du calcul du rendu (et des fps).
    \item [DoImageTask] est la task contenant les calculs du rendu. Voir \nameref{UsageTask} pour l'usage des taches.
\end{description}



\begin{figure}[h]
   \begin{center}
       \includegraphics[scale=0.5]{img/render.javafx/diagClassRender.png}
   \end{center}
   \caption{Diagramme du package render}
\end{figure}




\section{Détail des parties techniques}
    \subsection{Le Ray Tracer}
        \subsubsection{L'algorithme de base}
            \label{rayTracingBase}

            \subfile{latex/RT/algoBase.tex}

        \subsubsection{L'ombrage de Phong}
            \label{ombragePhong}

            \subfile{latex/RT/ombragePhong.tex}

        \subsubsection{Les réflexions}

            \subfile{latex/RT/reflections.tex}

        \subsubsection{Les réfractions}
            \label{refractions}

        \subsubsection{Le damier et le ciel (skybox)}
            \label{checkerboardSkybox}

            \subfile{latex/RT/checkerboardSkybox.tex}

    \subsection{Les formes}
	\subfile{latex/formes/geometryFormes.tex}

    \subsection{Le multithreading}

        \subfile{latex/RT/multithreading.tex}

    \subsection{Les mouvements de caméra}
        \label{mouvementsCamera}

        \subfile{latex/RT/mouvementsCam.tex}

    \subsection{L'interface graphique}

    \subsubsection{Pourquoi JavaFX ?}

Une question que nous nous sommes posées est : Swing ou JavaFX ?

Nous avons choisi d'utiliser JavaFX pour une principale raison. JavaFX est désormais la librairie officielle de Java. Swing n'étant plus maintenu, nous ne voulions pas développer avec une librairie qui commence à être dépréciée.

De plus, le sujet de Bataille Navale étant à faire obligatoirement en swing, nous voulions essayer les deux librairies.


\subsubsection{L'interface graphique}

Le développement de l'interface graphique a progressé au rythme du reste du projet, implémentant ce qui a été codé en parallèle. Plusieurs optimisations ont cependants été réalisés pour que l'affichage à l'écran impacte le moins possible les performances générales.

L'optimisation la plus impactante est l'usage d'un IntBuffer et de PixelFormat<IntBuffer>. Ils ont permis d'utiliser la méthode setPixels de PixelWriter, permettant ainsi de définir les pixel de la fenêtre de rendu d'un coup et non pas pixel par pixel comme fait précédement.

Un des effets secondaires engendré par les calculs lourds du rendu est la fenêtre qui n'était plus réactive.
Au départ, nous utililisions CameraTimer et WindowTimer, deux classes héritant de Animation Timer. WindowTimer lançait le rendu et CameraTimer s'occupait de gérér un déplacement de caméra. La méthode handle de AnimationTimer s'execute à chaque frame, ce qui nous permettait de faire les fps, la caméra et le rendu à l'aide de la méthode handle et tout était donc synchronisé.

Le problème posé par ce choix était que JavaFX gérait directement les calculs de rendus. La conséquencé était une interface graphique réactive à la vitesse du rendu (quelques images par secondes).

Pour palier à ce problème nous avons utiliser des classes proposés par le package concurrent de JavaFX. Le problème a été résolu grâce à la classe tasks de ce package, permettant de faire des calculs lourds sans impacter la réactivité de l'interface graphique. Comme javafx et le calcul de rendu n'étaient plus liés il a fallut synchroniser les mouvements de caméra et le calcul de l'image d'après.

\subsubsection{Le CSS}

En quelques mots, JavaFX permet de lier le style de ces fenêtre à des fichiers CSS. Nous en avons profitez pour aérer les fenêtres de choix de taille et la "toolbox".

\subsubsection{Le mode automatique}

Le mode automatique récupère la taille de l'écran principal, maximize la fenêtre et étire le rendu de manière à ce qu'il soit pixélisé mais qu'il occupe tout l'écran.

Dans une version plus ancienne la taille du rendu était également fixé par la taille de la fenêtre. Suite à l'ajout de la réfraction et de l'uvsphère qui sont assez couteuses en ressource nous avons décider de supprimer cette fonctionnalité.

\subsubsection{le conteur de fps}

Le conteur de FPS tire partie des avantages offerts par la classe AnimationTimer et utilise l'horodatage fournit par celle-ci pour calculer la vitesse de rendu.


    \subsection{Le parser}
        \subsubsection{Structure d'un fichier POV}
        \subsubsection{Implémentation du partern state}
        \subsubsection{Implémentation du patern template méthode}
        \subsubsection{Automate à état fini}


    \subsection{Les formes géométriques}




\section{Analyse des résultats}
    \subsection{Mesure des performances}
        \subfile{latex/mesuresPerf.tex}

    \subsection{Comparaison avec povRay}
\section{Idées d'améliorations}
    \subfile{latex/ideesAmelios.tex}
\section{Conclusion}






\newpage%Nouvelle page pour les annexes
\subfile{latex/appendices}

\newpage%Nouvelle page pour la bibliographie
\nocite{*}
\bibliographystyle{unsrt}
\bibliography{bibliography/sources}

\end{document}

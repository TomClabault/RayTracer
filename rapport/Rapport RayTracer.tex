\documentclass[11pt]{article}
\usepackage [french]{babel}
\usepackage [T1]{fontenc}

\usepackage[linesnumbered, ruled, french, onelanguage]{algorithm2e}
\usepackage{adjustbox}%Permet de centrer les figures dans la largeur de la page même si les figures sont plus larges que \textwidth
\usepackage{amssymb}
\usepackage{amsmath}
\usepackage[toc,page,title,titletoc,header]{appendix}
\usepackage{expl3}%Pour la control sequence /ExlpSyntaxOn demandée par l'utilisation de subfiles apparemment...
\usepackage{gensymb}%pour pouvoir écrire le signe °
\usepackage{geometry}%Pour changer la largeur des marges du document notamment
\usepackage{graphicx}
\usepackage{hyperref}%pour les liens dans la bibliographie
\usepackage{listings}
\usepackage{placeins}%pour utiliser FloatBarrier afin que les figure respectent bien leur position dans le code
\usepackage{slashbox}%Case séparée en deux tout en haut à gauche des tableaux à double entrées
\usepackage{stmaryrd}%pour les crochets à double barres d'intervalles de nombre entiers
\usepackage{tikz}
\usepackage{xcolor}%/definecolor et /color
\usepackage{subfiles}

\usepackage{etoolbox}%pour /AtBeginEnvironment
\AtBeginEnvironment{appendices}{\renewcommand{\thesection}{\Alph {section}}}%Pour recommencer à compter les sections à 0 en rentrant dans l'annexe et pour compter avec des lettres et non des chiffres
\renewcommand{\appendixpagename}{\centering Annexes}%Pour centrer le titre de la partie annexe
\renewcommand{\appendixtocname}{Table des annexes} % Pour faire apparaître les annexes dans la table of contents

%%%%%%%%%%%%%% Couleurs de: https://texblog.org/2011/06/11/latex-syntax-highlighting-examples/ %%%%%%%%%%%%%%
\definecolor{javared}{rgb}{0.6,0,0} % for strings
\definecolor{javagreen}{rgb}{0.25,0.5,0.35} % comments
\definecolor{javapurple}{rgb}{0.5,0,0.35} % keywords
\definecolor{javadocblue}{rgb}{0.25,0.35,0.75} % javadoc

\lstset
{
language=Java,
keywordstyle=\color{javapurple}\bfseries,
stringstyle=\color{javared},
commentstyle=\color{javagreen},
morecomment=[s][\color{javadocblue}]{/**}{*/},
numbers=left,
numberstyle=\tiny\color{black},
stepnumber=1,
numbersep=10pt,
tabsize=4,
showspaces=false,
showstringspaces=false}
%%%%%%%%%%%%%% Couleurs de: https://texblog.org/2011/06/11/latex-syntax-highlighting-examples/ %%%%%%%%%%%%%%





\author{Baptiste CLOCHARD \\
    Tom CLABAULT\\
    Thibaut LETOURNEUR\\
    Willie TANZIL}
\title{Ray-Tracer}
\date{}
\geometry{hmargin=3cm, vmargin=2cm}

\begin{document}
\maketitle
\newpage
\tableofcontents
\newpage

\maketitle

\section{Introduction}

    \subsection{Présentation du sujet}
    \subsection{Forme finale du projet}
    \subsection{Répartition du travail}

%\section{Mode d'emploi}
%\subsection{L'usage des scripts}

\subsection{L'utilisation de l'interface graphique}

\subsubsection{La fenêtre de choix de taille du rendu}

Lors de l'exécution du main de MainApp on obtient la fenêtre suivante :
\begin{figure}[h]
   \caption{Fênetre de sélection de la taille du rendu}
   \begin{center}
       \includegraphics{img/render.javafx/choiceWindow.jpg}
   \end{center}
\end{figure}

Cette fenêtre permet à l'utilisateur de sélectionner la résolution du rendu qu'il souhaite.
Activer le mode automatique maximisera la fenêtre et étendra le rendu sur tout l'écran. Si la résolution sélectionnée est alors inférieure à la résolution de l'écran, la qualité de l'image pourrait s'en trouver altérée.

\subsubsection{La fenêtre d'affichage du rendu}

Une fois que l'utilisateur a entré des paramètres valides, le rendu s'affichera.

Deux fenêtres s'afficheront alors :

\begin{figure}[h]
   \caption{Fenêtre du rendu}
   \begin{center}
       \includegraphics{img/render.javafx/render.jpg}
   \end{center}
\end{figure}

\begin{figure}[h]
   \caption{Fenêtre de la boite à outils}
   \begin{center}
       \includegraphics{img/render.javafx/toolbox.jpg}
   \end{center}
\end{figure}

%a déplacer dans readMe.txt

\section{Structuration du projet}
    \subsection{Les packages}
        \subsubsection{Le parser}
        	\subfile{latex/packages/parser_package.tex}
        \subsubsection{Liés au Ray Tracer}
            \subfile{latex/packages/RTpackages.tex}

        \subsubsection{Le multithreading}
	  \subfile{latex/packages/multithreading.tex}

        \subsubsection{Le package materials}
            \subfile{latex/packages/materials.tex}

        \subsection{Le package render}



\section{Détail des parties techniques}
    \subsection{Le Ray Tracer}
        \subsubsection{L'algorithme de base}
            \label{rayTracingBase}

            \subfile{latex/RT/algoBase.tex}

        \subsubsection{L'ombrage de Phong}
            \label{ombragePhong}

            \subfile{latex/RT/ombragePhong.tex}

        \subsubsection{Les réflexions}

            \subfile{latex/RT/reflections.tex}

        \subsubsection{Les réfractions}
            \label{refractions}

        \subsubsection{Le damier et le ciel (skybox)}
            \label{checkerboardSkybox}

            \subfile{latex/RT/checkerboardSkybox.tex}

    \subsection{Le multithreading}

        \subfile{latex/RT/multithreading.tex}

    \subsection{Les mouvements de caméra}
        \label{mouvementsCamera}

        \subfile{latex/RT/mouvementsCam.tex}

    \subsection{L'interface graphique}

    \subsubsection{Pourquoi JavaFX ?}

Une question que nous nous sommes posées est : Swing ou JavaFX ?

Nous avons choisi d'utiliser JavaFX pour une principale raison. JavaFX est désormais la librairie officielle de Java. Swing n'étant plus maintenu, nous ne voulions pas développer avec une librairie qui commence à être dépréciée.

De plus, le sujet de Bataille Navale étant à faire obligatoirement en swing, nous voulions essayer les deux librairies.


\subsubsection{L'interface graphique}

Le développement de l'interface graphique a progressé au rythme du reste du projet, implémentant ce qui a été codé en parallèle. Plusieurs optimisations ont cependants été réalisés pour que l'affichage à l'écran impacte le moins possible les performances générales.

L'optimisation la plus impactante est l'usage d'un IntBuffer et de PixelFormat<IntBuffer>. Ils ont permis d'utiliser la méthode setPixels de PixelWriter, permettant ainsi de définir les pixel de la fenêtre de rendu d'un coup et non pas pixel par pixel comme fait précédement.

Un des effets secondaires engendré par les calculs lourds du rendu est la fenêtre qui n'était plus réactive.
Au départ, nous utililisions CameraTimer et WindowTimer, deux classes héritant de Animation Timer. WindowTimer lançait le rendu et CameraTimer s'occupait de gérér un déplacement de caméra. La méthode handle de AnimationTimer s'execute à chaque frame, ce qui nous permettait de faire les fps, la caméra et le rendu à l'aide de la méthode handle et tout était donc synchronisé.

Le problème posé par ce choix était que JavaFX gérait directement les calculs de rendus. La conséquencé était une interface graphique réactive à la vitesse du rendu (quelques images par secondes).

Pour palier à ce problème nous avons utiliser des classes proposés par le package concurrent de JavaFX. Le problème a été résolu grâce à la classe tasks de ce package, permettant de faire des calculs lourds sans impacter la réactivité de l'interface graphique. Comme javafx et le calcul de rendu n'étaient plus liés il a fallut synchroniser les mouvements de caméra et le calcul de l'image d'après.

\subsubsection{Le CSS}

En quelques mots, JavaFX permet de lier le style de ces fenêtre à des fichiers CSS. Nous en avons profitez pour aérer les fenêtres de choix de taille et la "toolbox".

\subsubsection{Le mode automatique}

Le mode automatique récupère la taille de l'écran principal, maximize la fenêtre et étire le rendu de manière à ce qu'il soit pixélisé mais qu'il occupe tout l'écran.

Dans une version plus ancienne la taille du rendu était également fixé par la taille de la fenêtre. Suite à l'ajout de la réfraction et de l'uvsphère qui sont assez couteuses en ressource nous avons décider de supprimer cette fonctionnalité.

\subsubsection{le conteur de fps}

Le conteur de FPS tire partie des avantages offerts par la classe AnimationTimer et utilise l'horodatage fournit par celle-ci pour calculer la vitesse de rendu.


    \subsection{Le parser}
        \subsubsection{Structure d'un fichier POV}
        \subsubsection{Implémentation du partern state}
        	\subfile{latex/parser/pattern_state.tex}
        \subsubsection{Implémentation du patern template méthode}
        \subsubsection{Automate à état finit}


    \subsection{Les formes géométriques}




\section{Analyse des résultats}
    \subsection{Mesure des performances}
        \subfile{latex/other/mesuresPerf.tex}

    \subsection{Comparaison avec povRay}
\section{Idées d'améliorations}
TODO BUG PREMIERE TILE MUTLITHREADING
\section{Conclusion}






\newpage%Nouvelle page pour les annexes
\begin{appendices}
	\section{Le ray tracer}
		\subsection{Configuration d'une caméra, d'une scène et de rayons}
		\begin{figure}[h!]
			\adjustbox{center}{\includegraphics[width=1.1\textwidth]{img/rt/repCam2Rayonsv2.png}}

			\caption{Des rayons sont tirés depuis la caméra dans sa direction de regard. Nous cherchons les points d'intersection avec les objets de la scène}
		\end{figure}
		\FloatBarrier
		\label{annexe:repreCamRayon}

		\subsection{Le principe récursif des réflexions}
		\begin{figure}[!h]
			\adjustbox{center}{\includegraphics[width=1.1\textwidth]{img/rt/reflectionsSchema.png}}

			\caption{Exemple de rebonds successifs des rayons jusqu'à un objet non réfléchissant}
			\label{reflectionsSchema}
		\end{figure}
		\FloatBarrier
		\label{annexe:reflexionsRecursives}
\FloatBarrier
\end{appendices}

\newpage%Nouvelle page pour la bibliographie
\nocite{*}
\bibliographystyle{unsrt}
\bibliography{bibliography/sources}

\end{document}
